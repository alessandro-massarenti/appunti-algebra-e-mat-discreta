\chapter{Introduzione}

In generale non è consigliato acquistare nessun libro, però in esso sono contenuti molti esercizi e può quindi essere comodo.

\section{Argomenti di Algebra}

Questi argomenti prenderanno circa $\frac{2}{3}$ del corso.

\subsection{Congruenze e sistemi di Congruenze}
% da trasformare in esempio
Esempio:

\subsection{Matrici, operazioni sulle matrici, soluzioni di sistemi lineari}


%Magari da trasformare in una tabella quando avrò un po' di tempo

Valore 1 e si ci sarà nel primo parziale, argomento di algebra.



Si utilizzeranno le matrici per risolvere cose interessanti,

Le matrici sono tabelle rettangolari, a volte se ne usano di particolari a forma di quadrato.

Sulle matrici si impareranno le 3 operazioni, dove la terza è molto complessa e conta come 2, inoltre si impareranno altre 3 operazioni per passare da una matrice ad un'altra. 

In totale avremo 7 operazioni.

Le matrici ci serviranno a capire se un sistema anche enorme e che richiederebbe un lunghissimo calcolo ha soluzioni.

\subsection{Spazi vettoriali}

Valore 2

Si utilizzeranno somme di matrici, chiamate anche sovrapposizioni, le quali sono una generalizzazione di prodotto per numeri %e credo somma ma bisogna controllare

A fine corso questo argomento verrà applicato a modelli fisici.

\subsection{Diagonalizzazioni}

Valore 1 argomento di algebra

Posso diagonalizzare se posso scrivere prodotto di 3 quadrati, dove la matrice centrale è una matrice diagonale

\section{Argomenti di Matematica discreta}

Questi argomenti prenderanno circa $\frac{1}{3}$ del corso
\subsection{Grafi}

Valore 2

I grafi hanno notazione (V,E) dove V è il numero di vertici, ed E il numero di archi\footnote{V = vertex(Vertici), E = edges(archi)}.

esempio di grafo può essere dei villaggi su delle montagne, dove ogni arco rappresenta una strada che collega un villaggio, ed ogni villaggio è un vertice.

%Vedi esempio dalle slides
In questo esempio ci accorgiamo che un villaggio è isolato e un villaggio ha molti collegamenti.

\subsection{Metodi di conteggio}

Vale1 e sicuramente sarà presente al secondo pariale.
Questo argomento è molto dettagliato nel libro

Un'esempio sarà calcolare le diverse sequenze binarie (sequenze di \textit{zeri} ed \textit{uni})

Con 8 cifre dove abbiamo 6 uni e 2 zeri inizio a contare.

Ho 8 posizioni quindi prima sistemo gli zeri(Perchè sono meno)

%inserire disegno
E mi accorgo che il primo $0$ potrò metterlo in 8 posizioni ed il secondo in 7. Il numero di sequene sarà quindi uguale a $8\cdot7$

Però allo stesso tempo lo $0_1$ posso scambiarlo di posizione con lo $0_2$

Avrò quindi che il mio numero di sequenze sarà uguale a $\frac{8\cdot7}{2}$

\subsection{Relazioni di ricorrenza}

Vale 1 e anche questo è molto dettagliato sul libro.

Se devo calcolare una formula che riguarda n oggetti è una procedura che collega il saper calcolare per n-1 oggetti con il saper calcolare per n oggetti.

Ad esempio, se voglio calcolare il prodotto di n numeri naturali, ovvero il fattoriale, so che conoscendo il caso n-1 posso riutilizzarlo per calcolare il caso n.

Sarà però molto importante saper definire anche un caso base.

\[1\cdot2\cdot3\cdot... = n!\]

%Da rivedere un po'se è quello che intendeva
\[a_n \begin{cases}
    \cdot a_n -1 \\
    a_0 = 1 & \text{condizione base}

\end{cases}\]

Ci accorgiamo quindi che:

\begin{flalign*}
    a_1 =  1 \cdot a_0 = 1 \cdot 1 = 1 \\
    a_2 = 2 \cdot a_n
\end{flalign*}

