\chapter{Introduzione}
\section{Argomenti del corso}
\subsection{Congruenze e sistemi di Congruenze}
% da trasformare in esempio
Esempio:

\subsection{Matrici, operazioni sulle matrici, soluzioni di sistemi lineari}


%Magari da trasformare in una tabella quando avrò un po' di tempo

Valore 1 e si ci sarà nel primo parziale, argomento di algebra.



Si utilizzeranno le matrici per risolvere cose interessanti,

Le matrici sono tabelle rettangolari, a volte se ne usano di particolari a forma di quadrato.

Sulle matrici si impareranno le 3 operazioni, dove la terza è molto complessa e conta come 2, inoltre si impareranno altre 3 operazioni per passare da una matrice ad un'altra. 

In totale avremo 7 operazioni.

Le matrici ci serviranno a capire se un sistema anche enorme e che richiederebbe un lunghissimo calcolo ha soluzioni.

\subsection{Spazi vettoriali}

Valore 2

Si utilizzeranno somme di matrici, chiamate anche sovrapposizioni, le quali sono una generalizzazione di prodotto per numeri %e credo somma ma bisogna controllare

A fine corso questo argomento verrà applicato a modelli fisici.

\subsection{Diagonalizzazioni}

Valore 1 argomento di algebra

Posso diagonalizzare se posso scrivere prodotto di 3 quadrati, dove la matrice centrale è una matrice diagonale

\subsection{Grafi}

Valore 2

I grafi hanno notazione (V,E) dove V è il numero di vertici, ed E il numero di archi\footnote{V = vertex(Vertici), E = edges(archi)}.

esempio di grafo può essere dei villaggi su delle montagne, dove ogni arco rappresenta una strada che collega un villaggio, ed ogni villaggio è un vertice.

%Vedi esempio dalle slides
In questo esempio ci accorgiamo che un villaggio è isolato e un villaggio ha molti collegamenti.

\subsection{Metodi di conteggio}

Vale1 e sicuramente sarà presente al secondo pariale.

Un'esempio sarà calcolare le diverse sequenze binarie (sequenze di \textit{zeri} ed \textit{uni})

