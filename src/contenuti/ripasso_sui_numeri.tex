\chapter{Ripasso sui numeri}

\section{Insiemi di numeri}

$\left\{ 0,1,2,3,4,...\right\}$ ovvero \textit{se stesso + 1} è l'insieme $\mathbb{N}$ dei numeri naturali.

Possiamo fare la somma che ha come neutro 0

Ci accorggiamo subito che mangano gli opposti, dobbiamo però ampliare l'insieme ad un insieme li contenga.

$\left\{ ...,-4,-3,-2,-1,0,1,2,3,4,...\right\}$ si descrivono tramite la lettera $\mathbb{Z}$

Con questi numeri posso moltiplicare ma non posso dividere, se divido due interi non è detto ottenga un intero.

Per la moltiplicazione l'elemento neutro è l'$1$.