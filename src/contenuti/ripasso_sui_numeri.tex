\chapter{Ripasso sui numeri}

\section{Insiemi di numeri}

\paragraph{Numeri naturali}
$\left\{ 0,1,2,3,4,...\right\}$ ovvero \textit{se stesso + 1} è l'insieme $\mathbb{N}$ dei numeri naturali.

Possiamo fare la somma che ha come neutro 0

Ci accorggiamo subito che mangano gli opposti, dobbiamo però ampliare l'insieme ad un insieme li contenga.

\paragraph{Numeri interi}
$\left\{ ...,-4,-3,-2,-1,0,1,2,3,4,...\right\}$ si descrivono tramite la lettera $\mathbb{Z}$

Con questi numeri posso moltiplicare ma non posso dividere, se divido due interi non è detto ottenga un intero.

Per la moltiplicazione l'elemento neutro è l'$1$.

Non ci sono però gli inversi, avremo quindi bisogno di espandere l'insieme.

\paragraph{Numeri razionali} Questo insieme è l'insieme di tutte le possibili frazioni, possiamo definirli in notazione matematica come:

\[ \left\{ \frac{m}{n} \mid m,n\in \mathbb{Z} , n \neq 0 \right\} \]

Questo insieme però ha problemi di misura, useremo quindi degli altri numeri per evitare ogni problema

\paragraph{I numeri reali}

Per rappresentare i numeri reali si utilizza un sistema formato da una retta alla quale si fissa un sistema di coordinate ascisse. Ovvero fisso origine(che indicherò con 0), verso(Che solitamente va da sinistra verso destra), e unità di misura.

Per noi i numeri reali saranno tutti quei numeri che corrispondono a dei punti di una retta cha abbia queste coordinate.

%Ricordarsi di aggiungere la retta

Un esempio: Se ho un quadrato di lato 1 la diagonale, calcolata grazie al teorema di pitagora risulta essere $\sqrt{2}$. Questo risultato è un numero irrazionale.

Un altro esempio interessante ed utile è $\pi$, anche lui un numero razionale che risulta molto utile nel calcolo di circonferenze e formule afferenti.

NB I numeri reali riempiono completamente la retta.

\paragraph{Recap}
In questo momento la situazione degli insiemi è la seguente: $ \mathbb{N} \in \mathbb{Z} \in \mathbb{Q} \in \mathbb{R}$

\section{Problemi di soluzioni delle equazioni}

In algebra abbiamo un problema di soluzione delle equazioni.

\[ x-1 = 0 \Rightarrow x = 1 \]
\begin{align*}
    x^2 -1 &= 0 \\
    &= (x-1)(x+1)
\end{align*}

NB sarà molto importante saper fattorizzare bene.

Ci è molto utile capire che $x$ avrà tutte le soluzioni di $(x-1)$ e di $(x+1)$ ovvero due soluzioni.

altri esempi,

\[ x+2x+1 = 0 \Rightarrow \text{Due soluzioni coincidenti}\]

$x^2 +1  = 0 $ il risultato sarà sempre un risultato maggiore di $0$ ma nessuna soluzione nei reali. So quindi con certezza che ci sono due soluzioni coincidenti ma non so ancora quali sono\footnote{Vedremo più avanti}.

\section{I numeri complessi}

Per risolvere le equazioni descritte sopra amplio l'insieme in cui stiamo lavorando, ovvero quello dei numeri reali, e utilizzo i numeri complessi. Questi numeri si rappresentano con la lettera $\mathbb{C}$

Con i numeri complessi si introduce un nuovo numero, più precisamente una nuova unità. Questa si chiama unità immaginaria e si definisce con il simbolo $i$

\[ \exists i \mid i^2 = -1 \]

$i$ ora è soluzione dell'equazione $x^2 + 1 = 0$. Più precisamente le soluzioni sono $ i, -i$

%Importante da controllare negli appunti della prof un'equazione che andrebbe qui

Un numero complesso è un numero del tipo $z = a+ib \mid a,b \in \mathbb{R}$\footnote{Questa è la forma algebrica di $z$}

È importante descrivere che $a$ è parte reale di $z$ e $b$ è parte immaginaria di $z$.

Un esempio: $2-\sqrt{3}i$ in questa equazione $2$ è parte reale, la parte immaginaria è un numero meno $\sqrt{3}$

NB Tutti i numeri reali sono una forma particolare di numeri complessi.

\[ \left\{a+ib \mid a,b \in \mathbb{C} \right\} \]

\subsection{Somma e prodotto in $\mathbb{C}$}

\begin{equation*}
    z = 2 +3i
    w = 4-i
\end{equation*}
